\documentclass{bjfuthesis}
\usepackage[utf8]{inputenc}
\begin{document}
	\frontmatter
	\chapter*{摘要}
\begin{abstract}
	bjfuthesis文档类是为北京林业大学学生书写毕业论文准备的\LaTeX 文档类。该文档类基于ctexbook文档类。它修改了一些布局与样式来提供与北京林业大学毕业论文规范兼容的文档布局。
\end{abstract}
\keywordscn{LaTeX,毕业论文,北京林业大学}
\chapter*{Abstract}
\begin{abstract}
	The bjfuthesis class is a \LaTeX\space class intended for students in Beijing Forestry University to write their theses. The class is based on the ctexbook class. It modifies some of the layout and style definitions of these packages in order to provide a document layout that should be compatible with Beijing Forestry University Thesis Standard.
\end{abstract}
\keywordsen{LaTeX, thesis, Beijing Forestry University}
	\tableofcontents
	\mainmatter
	\chapter{绪论}
\section{研究背景}
国内诸多高校都在GitHub上提供了其高校的论文模板,故本作者秉持开源共享的精神,将本\LaTeX 文档类开源于https://github.com/bjfu-projects/bjfu-thesis。
\section{文档结构}
第一章为绪论,简单介绍研究背景以及文档结构。

第二章为英文使用说明,介绍本文档类的使用方法。

第三章为中文使用说明,介绍本文档类的使用方法。
\chapter{Usage Notes}
The order of following usage notes is based on the structure of the thesis.
\begin{enumerate}
\item To include the cover of your thesis, put `cover.pdf' in the directory.
\item You MUST add \verb|\frontmatter| before your abstract.
\item Print title of Chinese abstract or English abstract: \verb|\chapter*{abstract title}|
\item Print content of abstract: \verb|\begin{abstract}|abstract content\verb|\end{abstract}|
\item Print Chinese keywords: \verb|\keywordscn{Chinese keywords}|
\item Print English keywords: \verb|\keywordsen{English keywords}|
\item Print table of contents: \verb|tableofcontents|
\item You MUST add \verb|\mainmatter| before the main body of your thesis.
\item The hierachy of the thesis: 
\begin{verbatim}
	\chapter{chapter title}
	\section{section title}
	\subsection{subsection title}
\end{verbatim}
\item Use following structure:
\begin{verbatim}
\chapter{chapter 1 title}
normal text
\section{section 1.1 title}
normal text
\section{section 1.2 title}
\subsection{subsection 1.2.1 title}
normal text
...
\end{verbatim}
\item Floats are set to be centered by default, with default allowed positions being current position in code, top, bottom and a separate page.
\item Put images you want to insert in \verb|\figures|, and you DO NOT need to include 'figures' in your path
\item Show two captions for a figure or a table:\\ \verb|\bicaption{Chinese caption}{English caption}|
You MUST place figures in 'figure' environment, and table in 'table' environment, which MUST match, 
or \verb|\bicaption{Chinese caption}{English caption}| will not work as expected.
\item Show a note for a figure or table: \verb|\chartnote{a note for a figure or a table}|
\item You MUST add \verb|\backmatter| before your references and your acknowledgement.
\item Print bibliography: \verb|\printbibliography|
\end{enumerate}
\chapter{使用说明}
以下使用说明的顺序基于论文的结构
\begin{enumerate}
\item 要将你的论文封面插入论文,需将``cover.pdf"置于该目录中
\item 你必须在摘要部分前加入\verb|\frontmatter|
\item 显示中文或英文摘要标题:\verb|\chapter*{摘要标题}|
\item 显示摘要内容:\verb|\begin{abstract}|摘要内容\verb|\end{abstract}|
\item 显示中文关键词:\verb|\keywordscn{中文关键词}|
\item 显示英文关键词:\verb|\keywordsen{英文关键词}|
\item 显示目录:\verb|tableofcontents|
\item 你必须在你的论文的主体部分前加入\verb|\mainmatter|
\item 论文具有的层次划分:
\begin{verbatim}
	\chapter{一级标题}
	\section{二级标题}
	\subsection{三级标题}
\end{verbatim}
\item 使用以下层次结构:
\begin{verbatim}
	\chapter{一级标题 1}
	正文文字
	\section{二级标题 1.1}
	正文文字
	\section{二级标题 1.2}
	\subsection{三级标题 1.2.1}
	正文文字
	……
\end{verbatim}
\item 浮动体已默认居中,且默认允许的位置已设为“当前代码中的位置、顶部、底部、单独成页”
\item 将你需要插入的图片置于\verb|\figures|中,无需在路径中包含“figures”
\item 为一张图或一个表显示两个图(表)题:\verb|\bicaption{中文图题}{英文图题}|
你必须将图置于“figure”环境中,表置于“table”环境中,必须匹配。
否则\verb|\bicaption{中文图题}{英文图题}|不会起作用
\item 显示图表注: \verb|\chartnote{图的注释或表的注释}|
\item 你必须在你的参考文献前加入\verb|\backmatter|
\item 显示参考文献:\verb|\printbibliography|
\end{enumerate}
	\backmatter
    \chapter{参考文献}
	\chapter{致谢}
感谢所有为本项目提供过帮助的人的贡献。
\end{document}
